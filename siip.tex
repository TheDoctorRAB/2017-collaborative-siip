%        File: siip.tex
%     Created: Mon Feb 27 08:00 AM 2017 C
% Last Change: Mon Feb 27 08:00 AM 2017 C
%
\documentclass[11pt]{article}

\usepackage[acronym,toc]{glossaries}
\include{acros}
\makeglossaries
\usepackage{fancyhdr}
\usepackage{pagecounting}
\usepackage[dvips]{color}
\usepackage{graphicx}

% Trying to bold my name in the bib
\usepackage{xstring}
\def\FormatName#1{%
          \IfSubStr{#1}{Huff}{\textbf{#1}}{#1}%
          }

          \usepackage[left=1in, right=1in, top=1in, bottom=1in]{geometry}
          \newcommand\bb[1]{\mbox{\em #1}}
          \def\baselinestretch{1.1}
          %\pagestyle{empty}
          \newcommand{\hsp}{\hspace*{\parindent}}
          \definecolor{gray}{rgb}{0.4,0.4,0.4}

          \newcommand{\authorname}{Kathryn~D.~Huff }
          \newcommand{\authoremail}{katyhuff@illinois.edu}
          \newcommand{\authorsite}{arfc.npre.illinois.edu}

          \begin{document}

          \pagestyle{fancy}
          %\pagenumbering{gobble}
          %\fancyhead[location]{text}
          % Leave Left and Right Header empty.
          %\lhead{}
          %\rhead{}
          \lhead{\textcolor{gray}{Investigator: Prof. \authorname\\\authoremail}}
          \rhead{\textcolor{gray}{Advanced Reactors and Fuel Cycles\\Dept. of Nuclear, 
          Plasma, and Radiological Engineering}}
          %\rhead{\textcolor{gray}{\thepage/\totalpages{}}}
          \renewcommand{\headrulewidth}{0pt}
          \renewcommand{\footrulewidth}{0pt}
          \fancyfoot[C]{\footnotesize \textcolor{gray}{\authorsite}}

          \section{The Problem}
          As you read this, thousands of professors in the United States are 
          simultaneously preparing lessons on solving a conduction equation. They are doing so largely 
          without recieving feedback from one another or directly building on 
          one another's experience \cite{wilson_software_2014}.  It's distressing but 
          true that professors spend 10-40\% of 
          their time each year repeating curriculum development efforts already 
          tackled by numerous colleagues within their disciplines. What's worse 
          is that these efforts are rarely, if ever, reviewed by, shared with, 
          or extended upon by others.

          \section{The Proposed Solution}
          We propose a proof-of-concept for collaborative, open source, 
          curriculum development that will break down barriers between 
          classrooms and improve the transfer of lessons learned across universities.
          
          The goal will be to explore whether curriculum development can 
          operate like open source software development does, drawing on a 
          community of experts toward creating a shared core resource. In open 
          source software, developers share code revisions in online 
          repositories. In a typical project, the main 'fork' holds the 
          official copy of the software package. Developers each have their own 
          forks, copies of the repository, where they are able to make changes 
          on their own. 

          This initiative already involves a 
          few peer professors -- faculty in Nuclear Engineering from UC 
          Berkeley, U. Wisconsin at Madison, Kansas State, U. South Carolina, 
          and U. Tennessee-Knoxville. We all use GitHub \cite{github} to store, 
          revise, and collaborate on research, especially source code. A few 
          of us have already started to host our course curricula online as 
          well, but these are typically single author repositories \cite{}.

          The participants will collaborate on a master set of learning 
          modules} for a common upper-division course in nuclear engineering : 
          The Nuclear Fuel Cycle.  The curriculum will be hosted on GitHub, 
          used by all of us, and improved continually as we learn and grow as 
          instructors. 



          \paragraph{Each learning module} may include active in-class exercises, 
          presentation notes, homework problems, project descriptions, and 
          evaluation tools. 
          
          However, curriculum is not the same as source code, because each 
          instructor wants and deserves to impart their own style of teaching 
          (and corny jokes) on their instruction. This challenge has been 
          encountered by collaborative instruction organizations such as 
          Software Carpentry, in which forks of the original material diverge 
          without any strong incentive for contribution back to the master 
          material \cite{wilson_software_2014}.

          This proposal suggests that fine-grained modularity and a clear 
          dependency graph may assist in overcoming this goal. Further, a small 
          and devoted group of like-minded nuclear engineering professors 
          should be able to compromise on a cohesive set of modular materials 
          that can be mixed and matched according to individual preference. 
          This work could provide an important proof of concept for groups of 
          instructors willing to collaborate on open source curriculum for:

          - core courses with many sections in a single university
          - niche courses taught by a select group of professors across 
          universities
          - fundamental courses in small fields (like nuclear engineering)


          Why do educators not diff, patch, and merge one another's . 
          Collaboratively construct and improve lesson materials. 
         
          Scientists are more than happy to build upon one another's work, form 
          collaborations with others in their field. But, 
          when it comes to educating, where is the sharing of lessons learned 
          and collaboration? 

          \section{The Potential Impact}
          \cite{wilson_software_2014}.

          Doing code review at the end of the work isn't useful. What works is 
          incremental code review. This proposal suspects that the same is true 
          for curriculum review. 
          \cite{wilson_software_2014}.

          Education is an inherently distributed system, but this need not be a 
          hinderance to collaboration.

          \bibliographystyle{katyunsrt}
          \bibliography{bibliography}

          \end{document}


